% Options for packages loaded elsewhere
\PassOptionsToPackage{unicode}{hyperref}
\PassOptionsToPackage{hyphens}{url}
\PassOptionsToPackage{dvipsnames,svgnames,x11names}{xcolor}
%
\documentclass[
  12pt,
  a4paper,
]{article}

\usepackage{amsmath,amssymb}
\usepackage{iftex}
\ifPDFTeX
  \usepackage[T1]{fontenc}
  \usepackage[utf8]{inputenc}
  \usepackage{textcomp} % provide euro and other symbols
\else % if luatex or xetex
  \usepackage{unicode-math}
  \defaultfontfeatures{Scale=MatchLowercase}
  \defaultfontfeatures[\rmfamily]{Ligatures=TeX,Scale=1}
\fi
\usepackage[]{mathptmx}
\ifPDFTeX\else  
    % xetex/luatex font selection
\fi
% Use upquote if available, for straight quotes in verbatim environments
\IfFileExists{upquote.sty}{\usepackage{upquote}}{}
\IfFileExists{microtype.sty}{% use microtype if available
  \usepackage[]{microtype}
  \UseMicrotypeSet[protrusion]{basicmath} % disable protrusion for tt fonts
}{}
\makeatletter
\@ifundefined{KOMAClassName}{% if non-KOMA class
  \IfFileExists{parskip.sty}{%
    \usepackage{parskip}
  }{% else
    \setlength{\parindent}{0pt}
    \setlength{\parskip}{6pt plus 2pt minus 1pt}}
}{% if KOMA class
  \KOMAoptions{parskip=half}}
\makeatother
\usepackage{xcolor}
\usepackage[top=25mm,left=25mm]{geometry}
\ifLuaTeX
  \usepackage{luacolor}
  \usepackage[soul]{lua-ul}
\else
  \usepackage{soul}
  
\fi
\setlength{\emergencystretch}{3em} % prevent overfull lines
\setcounter{secnumdepth}{5}
% Make \paragraph and \subparagraph free-standing
\makeatletter
\ifx\paragraph\undefined\else
  \let\oldparagraph\paragraph
  \renewcommand{\paragraph}{
    \@ifstar
      \xxxParagraphStar
      \xxxParagraphNoStar
  }
  \newcommand{\xxxParagraphStar}[1]{\oldparagraph*{#1}\mbox{}}
  \newcommand{\xxxParagraphNoStar}[1]{\oldparagraph{#1}\mbox{}}
\fi
\ifx\subparagraph\undefined\else
  \let\oldsubparagraph\subparagraph
  \renewcommand{\subparagraph}{
    \@ifstar
      \xxxSubParagraphStar
      \xxxSubParagraphNoStar
  }
  \newcommand{\xxxSubParagraphStar}[1]{\oldsubparagraph*{#1}\mbox{}}
  \newcommand{\xxxSubParagraphNoStar}[1]{\oldsubparagraph{#1}\mbox{}}
\fi
\makeatother


\providecommand{\tightlist}{%
  \setlength{\itemsep}{0pt}\setlength{\parskip}{0pt}}\usepackage{longtable,booktabs,array}
\usepackage{calc} % for calculating minipage widths
% Correct order of tables after \paragraph or \subparagraph
\usepackage{etoolbox}
\makeatletter
\patchcmd\longtable{\par}{\if@noskipsec\mbox{}\fi\par}{}{}
\makeatother
% Allow footnotes in longtable head/foot
\IfFileExists{footnotehyper.sty}{\usepackage{footnotehyper}}{\usepackage{footnote}}
\makesavenoteenv{longtable}
\usepackage{graphicx}
\makeatletter
\def\maxwidth{\ifdim\Gin@nat@width>\linewidth\linewidth\else\Gin@nat@width\fi}
\def\maxheight{\ifdim\Gin@nat@height>\textheight\textheight\else\Gin@nat@height\fi}
\makeatother
% Scale images if necessary, so that they will not overflow the page
% margins by default, and it is still possible to overwrite the defaults
% using explicit options in \includegraphics[width, height, ...]{}
\setkeys{Gin}{width=\maxwidth,height=\maxheight,keepaspectratio}
% Set default figure placement to htbp
\makeatletter
\def\fps@figure{htbp}
\makeatother
% definitions for citeproc citations
\NewDocumentCommand\citeproctext{}{}
\NewDocumentCommand\citeproc{mm}{%
  \begingroup\def\citeproctext{#2}\cite{#1}\endgroup}
\makeatletter
 % allow citations to break across lines
 \let\@cite@ofmt\@firstofone
 % avoid brackets around text for \cite:
 \def\@biblabel#1{}
 \def\@cite#1#2{{#1\if@tempswa , #2\fi}}
\makeatother
\newlength{\cslhangindent}
\setlength{\cslhangindent}{1.5em}
\newlength{\csllabelwidth}
\setlength{\csllabelwidth}{3em}
\newenvironment{CSLReferences}[2] % #1 hanging-indent, #2 entry-spacing
 {\begin{list}{}{%
  \setlength{\itemindent}{0pt}
  \setlength{\leftmargin}{0pt}
  \setlength{\parsep}{0pt}
  % turn on hanging indent if param 1 is 1
  \ifodd #1
   \setlength{\leftmargin}{\cslhangindent}
   \setlength{\itemindent}{-1\cslhangindent}
  \fi
  % set entry spacing
  \setlength{\itemsep}{#2\baselineskip}}}
 {\end{list}}
\usepackage{calc}
\newcommand{\CSLBlock}[1]{\hfill\break\parbox[t]{\linewidth}{\strut\ignorespaces#1\strut}}
\newcommand{\CSLLeftMargin}[1]{\parbox[t]{\csllabelwidth}{\strut#1\strut}}
\newcommand{\CSLRightInline}[1]{\parbox[t]{\linewidth - \csllabelwidth}{\strut#1\strut}}
\newcommand{\CSLIndent}[1]{\hspace{\cslhangindent}#1}

\usepackage{graphicx}
\makeatletter
\@ifpackageloaded{caption}{}{\usepackage{caption}}
\AtBeginDocument{%
\ifdefined\contentsname
  \renewcommand*\contentsname{Table of contents}
\else
  \newcommand\contentsname{Table of contents}
\fi
\ifdefined\listfigurename
  \renewcommand*\listfigurename{List of Figures}
\else
  \newcommand\listfigurename{List of Figures}
\fi
\ifdefined\listtablename
  \renewcommand*\listtablename{List of Tables}
\else
  \newcommand\listtablename{List of Tables}
\fi
\ifdefined\figurename
  \renewcommand*\figurename{Figure}
\else
  \newcommand\figurename{Figure}
\fi
\ifdefined\tablename
  \renewcommand*\tablename{Table}
\else
  \newcommand\tablename{Table}
\fi
}
\@ifpackageloaded{float}{}{\usepackage{float}}
\floatstyle{ruled}
\@ifundefined{c@chapter}{\newfloat{codelisting}{h}{lop}}{\newfloat{codelisting}{h}{lop}[chapter]}
\floatname{codelisting}{Listing}
\newcommand*\listoflistings{\listof{codelisting}{List of Listings}}
\makeatother
\makeatletter
\makeatother
\makeatletter
\@ifpackageloaded{caption}{}{\usepackage{caption}}
\@ifpackageloaded{subcaption}{}{\usepackage{subcaption}}
\makeatother

\ifLuaTeX
  \usepackage{selnolig}  % disable illegal ligatures
\fi
\usepackage{bookmark}

\IfFileExists{xurl.sty}{\usepackage{xurl}}{} % add URL line breaks if available
\urlstyle{same} % disable monospaced font for URLs
\hypersetup{
  pdftitle={The Economics of App Success: How Revenue Streams Influence Downloads and User Ratings},
  pdfauthor={Group 5},
  colorlinks=true,
  linkcolor={blue},
  filecolor={Maroon},
  citecolor={Blue},
  urlcolor={Blue},
  pdfcreator={LaTeX via pandoc}}


\title{\textbf{The Economics of App Success: How Revenue Streams
Influence Downloads and User Ratings}}
\usepackage{etoolbox}
\makeatletter
\providecommand{\subtitle}[1]{% add subtitle to \maketitle
  \apptocmd{\@title}{\par {\large #1 \par}}{}{}
}
\makeatother
\subtitle{Group Assignment - Strategy and Business Models}
\author{Group 5}
\date{28/11/2024}

\begin{document}
\maketitle

\vspace{2em}
\begin{center}
  \includegraphics[width=0.3\textwidth]{figures/jads.png}
\end{center}
\vspace{3em}
\newpage

\renewcommand*\contentsname{Contents}
{
\hypersetup{linkcolor=}
\setcounter{tocdepth}{3}
\tableofcontents
}

\newpage

\section{\texorpdfstring{Introduction
}{Introduction }}\label{introduction}

With the every-growing popularity of cellphones (Richter 2023), the
popularity of mobile applications is also steadily increasing. In 2024,
mobile applications are estimated to generate over \$900 billion in
revenue ({``Mobile App Revenue Worldwide by Segment (2019-2027)''}
2023). Generally, mobile applications (`\emph{apps}' from here on) tend
to be categorized in three different categories (Roma and Ragaglia
2016). Paid apps are the most transparent; they revenue is based on an
up-front purchase by the user. Free apps, on the other hand, require no
purchase by the user at any stage. According to Roma and Ragaglia
(2016), these apps make their revenue from deals with third-parties,
either through advertisement or other purposes such as market
information.

Finally, freemium apps are, as the name suggests, a middle-ground
between free and premium. Users get access to a basic version of the
application first and can unlock more features through an in-app payment
(Kumar 2014). Of these three revenue models, freemium is the most
commonly used and the most (Salehudin and Alpert 2021) and leads to more
downloads as well as revenue (Liu, Au, and Choi 2014).

\subsection{Academic Background}\label{academic-background}

Most research uses these three established categories---paid, freemium,
and free---when discussing revenue models for apps. However, by limiting
the discussion to these three terms, nuances within these categories
might be missed.

In a review paper from 2023 (Djaruma et al. 2023), different levels of
monetization are suggested based on previous literature. These levels
provide a clear framework for the revenue models of mobile apps.

\begin{longtable}[]{@{}
  >{\raggedright\arraybackslash}p{(\columnwidth - 2\tabcolsep) * \real{0.2857}}
  >{\raggedright\arraybackslash}p{(\columnwidth - 2\tabcolsep) * \real{0.7143}}@{}}
\toprule\noalign{}
\begin{minipage}[b]{\linewidth}\raggedright
Strategy
\end{minipage} & \begin{minipage}[b]{\linewidth}\raggedright
Description
\end{minipage} \\
\midrule\noalign{}
\endfirsthead
\toprule\noalign{}
\begin{minipage}[b]{\linewidth}\raggedright
Strategy
\end{minipage} & \begin{minipage}[b]{\linewidth}\raggedright
Description
\end{minipage} \\
\midrule\noalign{}
\endhead
\bottomrule\noalign{}
\endlastfoot
Level 5: Premium & Pay to use the application. This either happens
up-front, or after a trial period. \\
Level 4: Semi-premium & Use a limited number of features for free.
Unlock the app with all features through an in-app purchase. \\
Level 3: In-app advertisement and in-app purchases & Free application
with ads, encouraging users to remove ads or to make in-app
purchases. \\
Level 2: Sample and premium & Two different versions of the same app.
One is a version with limit features and/or ads. The other version is a
premium version. \\
Level 1: In-app advertisement & Only one version of the app, with only
ads and no in-app purchases. \\
Level 0: Free & The app has no monetization. However, money can still be
made through selling user information. \\
\caption{Six levels of monetization for
apps}\label{tbl-levels}\tabularnewline
\end{longtable}

Table~\ref{tbl-levels}

\subsection{Societal Background}\label{societal-background}

Currently, most apps utilize the freemium revenue model (Salehudin and
Alpert 2021). However, as discussed in Djaruma et al. (2023), there are
many revenue models between completely premium and completely free. A
more fine-grained classification of app revenue models beyond the
traditional ``paid-freemium-free'' framework holds significant societal
and business implications.

For society, such distinctions enhance transparency. Some monetization
models, such as free or ad-filled apps, may rely on selling user
information as a source of revenue (Bamberger et al. 2020). Therefore,
clearer distinctions regarding the revenue model will empower consumers
to make informed choices. It may also enable policymakers to identify
and regulate exploitative practices, such as manipulative
microtransactions or intrusive ad models, ensuring all applications
align with ethical and legal standards (Mileros and Forchheimer 2024).

For businesses, this paper should unlock more insight into the
effectiveness of different revenue streams. This will allow developers
to tailor monetization strategies to specific audiences. Furthermore,
both consumers and regulatory bodies are growing more concerned with the
privacy concerns of apps, especially ones that rely on market
information (Mileros and Forchheimer 2024). A granular understanding
helps businesses adapt, aligning profitability with sustainability and
ethical considerations.

\subsection{Research Gap}\label{research-gap}

In short, apps play an increasingly important role in our techno-centric
society. To improve the user experience and increase profits,
consideration of revenue models is key. Despite the great depth of
research on this topic, literature tends to be focussed on the three big
categories of paid, freemium, and free. This lack of nuance prevents us
from understanding the fine-grained details that may help improve future
apps.

The levels of monetization as proposed by Djaruma et al. (2023) would
allow for this nuance. However, their framework has never been used in
an empirical setting, as the paper by Djaruma et al. (2023) was
published only last year. Applying this framework to see how different
revenue streams impact the popularity of an app may yield valuable
insights into the preferences of consumers. Therefore, the question to
answer within this paper will be: \emph{How are the six different
revenue models as proposed by Djaruma et al. (2023) correlated to the
success of an app?}

\section{\texorpdfstring{Theory and Hypotheses
}{Theory and Hypotheses }}\label{theory-and-hypotheses}

In this section, prior research into the topic of revenue streams and
its correlation to success in apps will be discussed. As mentioned in
the Introduction section, this paper will apply the six levels of
revenue as proposed by Djaruma et al. (2023) to app data. The following
section will contain a holistic overview of the existing research, as
well as hypotheses that arise from this theoretical framework.

\subsection{Literature Review}\label{literature-review}

To answer the question ``\emph{How are the six different revenue models
as proposed by Djaruma et al. (2023) correlate to the success of an
app?}'', we must first define what constitutes to success. In this
paper, success will be defined by a couple of factors: popularity,
rating, and estimated revenue.

\subsubsection{Popularity}\label{popularity}

The popularity of an app can be measured by the number of downloads. It
is important to note the popularity of an app is complex, and is not
solely dependent on the chosen revenue model. Other features, such as
whether an app is featured on charts, whether it has frequent updates,
and word-of-mouth awareness, will also impact the popularity of an app
(Aydin Gokgoz, Ataman, and Bruggen 2021). However, despite these other
variables, to versions of the same app will still have drastically
different performances with different revenue streams (Liu, Au, and Choi
2014).

H1a: Apps that allow the user to have free access to all features (level
0 and 1) will have the highest amount of downloads overall. However, the
ratings may fluctuate, as quality can vary for free-to-access apps.

H1b: The apps with the most downloads will be level 1. Most social media
platforms, which dominate our culture, tend to have this revenue stream
(Djaruma et al. 2023).

H1c: For apps that utilize a sample and a premium version of the same
app (level 2), the free versions of an app will have more downloads than
their paid-for counterpart. Most, if not all, users will download the
free version first, and then might upgrade. This means there should be a
disparity between the number of downloads between the apps, as is also
demonstrated by Liu, Au, and Choi (2012).

H1d: The most downloaded apps in the gaming category will likely fall
under level 4. Many popular games use this type of ``pay-to-win''
mechanism (Nieborg 2016). Therefore, it would be expected this same
pattern would arise from our data.

\subsubsection{Rating}\label{rating}

The downloads of an app are not everything. An app can be downloaded
often, but may not be highly rated.

H2a: Apps that require the user to pay to unlock features (level 2, 3,
and 4) will tend to have lower ratings than the version that requires
payment upfront (level 5). The main draw of a freemium model is to
attract users, and have them update to a paid version (Kumar 2014).
However, as Kumar (2014) points out, this can be a double-edged sword.
Too few features, and it may not be attractive to users. Too many
features, and the users will not update.

H2b: Fully premium apps (level 5) will have less variance in their
ratings, while all other levels will have more. In the same vein as H2a,
users have more realistic expectations of paid apps compared to apps
that require you to unlock features (Kumar 2014). Therefore, more users
downloading premium apps will be satisfied with their purchase, leading
to less variance.

H2c: For apps that utilize a sample and a premium version of the same
app (level 2), the rating of the paid-for version is positively
associated with the rating of the free version of the same app. This was
true for the study on the most popular apps in the Google Play Store by
Liu, Au, and Choi (2012), so it is expected a similar pattern should
arise for this dataset.

\subsubsection{Revenue Estimation}\label{revenue-estimation}

It is important to point out downloads and ratings likely do not
directly correlate to the actual revenue of an app. The revenue of apps
``premium'' apps that require an upfront payment, the revenue is
relatively simple to track and compare. However, for apps that rely on
advertisement, in-app purchases and/or selling market information, this
is harder to track.

For apps that solely on advertisement, time retention can be a good
measure of revenue (Ross 2018). However, this only works if the app
solely relies on ads. An example of this given by Djaruma et al. (2023)
is TikTok: this app relies not only on advertisement, but also on users
purchasing products through its shop. Therefore, using solely the time
retention would not accurately capture the revenue of an app with both
revenue streams. Furthermore, the selling of user data is usually not
publicized, meaning it is not possible to know the revenue from this.

Unfortunately, our data only contains the price of ``premium'' app
versions. The data does not include any details regarding in-app
purchases nor time-retention. Because of this lack of sufficient data,
solely downloads and ratings will be taken into account as indicators of
success.

\section{\texorpdfstring{Methods and Data
}{Methods and Data }}\label{methods-and-data}

This section will further elaborate on the dataset and methods used to
test the hypotheses outlined in the previous section. The focus lies on
providing a comprehensive description of the dataset, including its
structure and the variables it contains, followed by an explanation of
the variable selection process. Additionally, the statistical methods
applied are briefly explained, combined with a explanation of how
assumptions, such as missing values and potential biases, were addressed
to ensure the robustness of our analysis.

\subsection{Dataset Description}\label{dataset-description}

The dataset used for this research consists of 1,016,666 instances and
27 variables, representing a detailed overview of mobile applications
across various revenue models. Each instance corresponds to an app, and
the variables capture key attributes such as app downloads, user
ratings, and monetization strategies. Below is an overview of some of
variables:

\begin{longtable}[]{@{}
  >{\raggedright\arraybackslash}p{(\columnwidth - 2\tabcolsep) * \real{0.2248}}
  >{\raggedright\arraybackslash}p{(\columnwidth - 2\tabcolsep) * \real{0.7752}}@{}}
\toprule\noalign{}
\begin{minipage}[b]{\linewidth}\raggedright
Variable
\end{minipage} & \begin{minipage}[b]{\linewidth}\raggedright
Description
\end{minipage} \\
\midrule\noalign{}
\endhead
\bottomrule\noalign{}
\endlastfoot
my\_app\_id (object) & Unique identifier for each app. \\
date\_published (object) & The publication date of the app. Only three
missing values (0.000295\% null). \\
privacy\_policy (object) & Information about the app's privacy policy,
missing in 28.57\% of cases. \\
rating\_app (float64) & The average rating of the app, with 8.76\%
missing values. \\
nb\_rating (object) & Number of ratings received by the app, missing in
8.76\% of cases. \\
num\_downloads (object) & The number of downloads for the app, nearly
complete with only 15 missing values (0.001475\% null). \\
price\_gplay (object) & The price of the app as listed on Google Play,
missing in 0.43\% of cases. \\
in\_app (bool) & Indicates whether the app has in-app purchases (no
missing values). \\
has\_ads (bool) & Indicates whether the app contains advertisements (no
missing values). \\
content\_rating\_app (object) & The app's content rating, with three
missing values (0.000295\% null). \\
developer\_name (object) & The name of the app developer, missing in
only 16 instances (0.001574\% null). \\
\end{longtable}

The dataset includes several additional variables related to app
features, developer information, and user engagement metrics such as
visit\_website, more\_from\_developer, and family\_library. However,
some variables, such as whats\_new (100\% null) and in\_app\_product
(89.57\% null), were deemed unsuitable for analysis due to their high
proportion of missing data.

The primary purpose of this dataset in this study, is to analyze app
monetization strategies by categorizing apps into distinct revenue
levels and evaluating their performance based on key metrics like
downloads and user ratings.

\subsection{Variable Selection}\label{variable-selection}

The dataset utilized in this study consists of 1,016,666 entries,
encompassing a broad range of attributes related to mobile applications.
For the purpose of our analysis, 13 variables were selected, capturing
critical information about app characteristics, user engagement,
monetization strategies, and developer details. These variables include
the app's unique identifier (my\_app\_id), the total number of downloads
(num\_downloads), average user ratings (rating\_app), and the number of
ratings (nb\_rating). Additionally, the dataset provides information on
app pricing (price\_gplay), the presence of in-app purchases (in\_app),
and whether the app includes advertisements (has\_ads). Other variables,
such as content ratings (content\_rating\_app), app categories
(categ\_app), and developer information (developer\_name and
developer\_info), further enhance the richness of the dataset. This
subset of variables allows us to comprehensively examine the interplay
between monetization strategies and app success.

To systematically explore monetization strategies, we classified the
apps into six distinct levels based on their monetization models. These
levels reflect varying approaches to generating revenue, ranging from
completely free apps to fully premium paid apps.

Level 0 represents apps with no monetization, offering free services
without ads or in-app purchases. At the opposite end, Level 5 includes
premium apps requiring upfront payment, free from ads or in-app
purchases, delivering a premium experience.

In between, Level 1 consists of free apps monetized solely through ads,
while Level 3 combines ads and in-app purchases, offering additional
features for users willing to pay. Level 4 refines the freemium model by
removing ads and relying entirely on in-app purchases to monetize.

Level 2 employs a dual-version strategy, featuring both free sample apps
with limited functionality (and potentially ads) and paid premium apps
with comprehensive features and no ads or in-app purchases.

This classification is grounded in theoretical frameworks, such as the
monetization levels proposed by Djaruma et al.~(2023) and the App
business models of (CITE), and allows for a nuanced analysis of how
different revenue models impact app success metrics like user ratings
and downloads. Our systematic categorization facilitates a deeper
understanding of the relationship between monetization strategies and
app performance.

\subsection{Statistical Methods}\label{statistical-methods}

To test our hypotheses, we employed a combination of descriptive
statistics, text processing, and machine learning techniques.
Descriptive statistics were utilized to analyze distributions and trends
in metrics such as num\_downloads, rating\_app, and price\_gplay. We
categorized applications into six monetization levels based on binary
indicators: is\_free, in\_app, and has\_ads. Price values were processed
to distinguish between free and paid applications.

To identify paired sample and premium applications within level 2, we
applied Term Frequency-Inverse Document Frequency (TF-IDF) vectorization
combined with cosine similarity on application names. This approach is
effective for measuring textual similarity between documents (CITE
Source 2). Subsequent filtering involved prefix matching and the
identification of indicative terms (e.g., ``Free,'' ``Pro'') to ensure
logical pairing based on naming conventions and shared developers.

To adjust ratings for applications with few reviews, we calculated a
Bayesian average. This method provides a more robust measure of user
satisfaction by accounting for the number of ratings and the overall
average rating across all applications (CITE Source 3). Visualizations,
including scatter plots and box plots, were employed to explore
relationships between monetization levels and user engagement metrics
which will be displayed in the results section.

\subsubsection{Handeling of Assumptions}\label{handeling-of-assumptions}

We addressed missing values by removing rows with critical nulls, such
as those in num\_downloads, to maintain data integrity. Text-based
variables like content\_rating\_app were standardized to ensure
consistency. For price\_gplay, currency symbols were removed to
facilitate the classification of applications into free or paid
categories.

Outliers in metrics like num\_downloads were retained if they
represented industry-leading applications, as their exclusion could skew
the analysis. The use of Bayesian averages mitigated bias in rating\_app
due to low review counts, providing a more accurate reflection of user
satisfaction. Covariance checks were conducted to ensure the absence of
multicollinearity among numerical variables, thereby enhancing the
reliability of correlation and regression analyses.

Some applications exhibited rare combinations of is\_free, in\_app, and
has\_ads that did not fit within the predefined monetization levels.
These applications were excluded from the analysis but documented as a
limitation. Edge cases in level 2 application pairing were flagged for
potential mismatches due to naming ambiguities, ensuring transparency in
the classification process.

These methodologies facilitated a systematic and accurate exploration of
monetization models and their impact on application performance.

\section{\texorpdfstring{Results }{Results }}\label{results}

In this section, we will visualize the data through tables and
visualizations. These plots largely explore the data around the
hypotheses and research question, we discussed in the previous sections.
The aim of this section is to present possible evidence in supporting a
hypothesis. This will be discussed and concluded upon in the next
section.

Before we look at the results, let's revisit the research question:
``\emph{How are the six different revenue models as proposed by Djaruma
et al. (2023) correlate to the success of an app?}''. Where we identify
6 different revenue models, described in this section as (revenue)
levels.

\begin{longtable}[]{@{}
  >{\raggedright\arraybackslash}p{(\columnwidth - 2\tabcolsep) * \real{0.2857}}
  >{\raggedright\arraybackslash}p{(\columnwidth - 2\tabcolsep) * \real{0.7143}}@{}}
\toprule\noalign{}
\begin{minipage}[b]{\linewidth}\raggedright
Strategy
\end{minipage} & \begin{minipage}[b]{\linewidth}\raggedright
Description
\end{minipage} \\
\midrule\noalign{}
\endhead
\bottomrule\noalign{}
\endlastfoot
Level 5: Premium & Pay to use the application. This either happens
up-front, or after a trial period. \\
Level 4: Semi-premium & Use a limited number of features for free.
Unlock the app with all features through an in-app purchase. \\
Level 3: In-app advertisement and in-app purchases & Free application
with ads, encouraging users to remove ads or to make in-app
purchases. \\
Level 2: Sample and premium & Two different versions of the same app.
One is a version with limit features and/or ads. The other version is a
premium version. \\
Level 1: In-app advertisement & Only one version of the app, with only
ads and no in-app purchases. \\
Level 0: Free & The app has no monetization. However, money can still be
made through selling user information. \\
\end{longtable}

The dataset is divided into these different levels. From this pie chart,
we can make up that more than 75\% of all the apps belong to level 0 and
1. With the smallest population being level 2 with two different version
of the same app.

To find the success of an app, we have established in the ``Literature
Review'' subsection, that it can be measured by:

\begin{enumerate}
\def\labelenumi{\arabic{enumi}.}
\tightlist
\item
  Popularity: which can be measured by the number of downloads.
\item
  The rating of an application.
\item
  \st{Revenue estimation} (cannot be used, due to lack of data).
\end{enumerate}

\subsection{Number of Downloads.}\label{number-of-downloads.}

The graphs below provides insights into the number of app downloads
categorized by revenue levels. The left graph displays the total amount
of downloads, while the right graph displays the average amount of
downloads. In the next following three subsections, we will look into
the key takeaways from these two graphs:

\begin{itemize}
\item
  Distribution of app downloads: how the downloads are distributed among
  the revenue levels.
\item
  Gaming apps: analyzing the importance of gaming apps.
\item
  Free vs Paid: comparing free and paid apps.
\end{itemize}

\subsubsection{Distribution of app
downloads}\label{distribution-of-app-downloads}

\textbf{Important takeaway:} Revenue levels 0 and 1 have a significant
drop-off in terms of average number of downloads.

This aligns well with hypothesis 1b which focusses in particular on
revenue level 1:

\emph{The apps with the most downloads will be level 1. Most social
media platforms, which dominate our culture, tend to have this revenue
stream (Djaruma et al. 2023).}

For instance in the the graph above, we see that the apps are highly
concentrated. So, a small proportion of the apps make up for the vast
majority of the downloads.

The graph below illustrates the distribution of app downloads per
revenue level, with download ranges highlighted using distinct colors.
The interesting takeaway is that level 0 and 1 have the most app with
over 1 billion downloads.

The graph below shows the number of apps with over 1 billion downloads
per category. The communication category stands out with the most apps
with over 1 billion downloads.

Taking a closer look at the apps provided by (Djaruma et al. 2023). We
do see that Facebook, Instagram, Spotify, Snapchat and Amazon Shopping
fall under level 1. (Disclaimer: TikTok didn't exist up until 2019)

\subsubsection{Gaming apps}\label{gaming-apps}

\textbf{Important takeaway:} Revenue level 3 and 4 stands out with high
average app downloads. Suggesting these revenue levels can be considered
as the most popular revenue level. Important to note is that they have
two things in common, they are free and contain in-app purchases. The
only difference is that level 3 contain ads.

This takeaway aligns well with hypothesis 1d, which takes a closer look
into these 2 revenue levels:

\emph{The most downloaded apps in the gaming category will likely fall
under level 4. Many popular games use this type of ``pay-to-win''
mechanism (Nieborg 2016). Therefore, it would be expected this same
pattern would arise from our data.}

Games are hugely popular on the Google Play Store. In the graphs below,
you'll find that they are responsible for more than 25\% of the total
downloads, while only populating 15\% of the total apps.

The table below is a comparison between the average downloads of gaming
apps and app with other categories. The gaming apps are therefore
downloaded more on average than apps from other categories, with nearly
100\% higher download average. This indicates that gaming apps are
hugely popular on the Google Play Store.

By filtering on the gaming categories we can see that the average
downloads is also significantly higher in level 3 and 4. With level 4
having a higher variance than level 3 and 4.

To further explore this, the table below shows the top 5 most downloaded
games from each revenue level. It reveals a noteable variance in
downloads between the 2 revenue levels. The variance in level 4 is much
larger than in level 3, even if it only displays the top 5. For
instance, the downloads in level 4 varies from 0.5 to 0.1 billion. While
the downloads in level 3 range from 1.0 to 0.5 billion.

\subsubsection{Free vs Paid}\label{free-vs-paid}

\textbf{Key takeaways:}

\begin{itemize}
\item
  Revenue level 2 premium and 5 have the least amount of total and
  average app downloads. These apps cost money upon downloading the app.
\item
  Revenue level 2 sample and premium have a high download offset in both
  average and total amounts.
\end{itemize}

We see high disparity in downloads between free (-mium) and paid apps.
To further investigate this we follow the guideline of hypotheses: 1a
and 1c.

\emph{H1a: Apps that allow the user to have free access to all features
(level 0 and 1) will have the highest amount of downloads overall.
However, the ratings may fluctuate, as quality can vary for
free-to-access apps.}

\emph{H1c: For apps that utilize a sample and a premium version of the
same app (level 2), the free versions of an app will have more downloads
than their paid-for counterpart. Most, if not all, users will download
the free version first, and then might upgrade. This means there should
be a disparity between the number of downloads between the apps, as is
also demonstrated by Liu, Au, and Choi (2012).}

The graph below illustrates that level 0 and 1 have indeed the highest
overall downloads, compared to the other levels combined. (Apps that
fall under level 0 and 1 have free access to all features.)

As show in figure 2, level 2 sample and premium have a high download
offset in both average and total amounts. Illustrating just how high
this offset is, the graph below illustrates that out of all the
downloaded apps in level 2, just 1.1\% are premium apps.

In terms of average downloads the disparity is also high. As can be seen
by the graph below.

However, by looking at the average downloads in the graph below, we see
that the average download difference is close to 500.000. With an
average download ratio of nearly 27\%. Meaning that about one in fourth
users that download the sample app, also download the premium app.

\subsection{Ratings}\label{ratings}

Plotting the distribution of the ratings across all the different
revenue models doesn't have the same insight as with the number of
downloads. As can be seen below, the ratings in itself doesn't vary all
that much within the revenue levels.

When plotting the variance of the ratings across the different levels,
we do see a lot of variance. However, this variance diminishes when
considering the Bayesian average, which smooths out the fluctuations and
provides a more consistent view of the ratings. Indicating that some
levels have strong outliers.

In the next two subsections we investigate further on the variance and
the difference in rating between paid and free apps.

\subsubsection{Quality of Apps}\label{quality-of-apps}

Hypotheses 1a and 2a partially touch on the quality of apps being a
reason for the fluctuations in ratings. While 1a looks at the free
access to all feature apps, 2a looks more at the drawbacks of the
freemium revenue model. Where the balance between free and paid features
can result in lower ratings.

\emph{H1a: Apps that allow the user to have free access to all features
(level 0 and 1) will have the highest amount of downloads overall.
However, the ratings may fluctuate, as quality can vary for
free-to-access apps.}

\emph{H2a: Apps that require the user to pay to unlock features (level
2, 3, and 4) will tend to have lower ratings than the version that
requires payment upfront (level 5). The main draw of a freemium model is
to attract users, and have them update to a paid version (Kumar 2014).
However, as Kumar (2014) points out, this can be a double-edged sword.
Too few features, and it may not be attractive to users. Too many
features, and the users will not update.}

The table below shows the variance and standard deviation between ``free
access to all feature apps'' and ``free access to not all features or
paid apps''. There is not a big statistical difference.

As can also be seen in the graph below, that shows the boxplot of the
ratings.

Conversely, the mean rating and Bayesian average of apps where you need
to unlock features versus paid apps with all features also doesn't see a
big difference. The premium apps do tend to have a higher rating. As can
be seen in the graph below.

\subsubsection{Variance in Ratings}\label{variance-in-ratings}

Hypothesis 2b touches on the variance between premium and their
counterpart freemium apps.

\emph{H2b: Fully premium apps (level 5) will have less variance in their
ratings, while all other levels will have more. In the same vein as H2a,
users have more realistic expectations of paid apps compared to apps
that require you to unlock features (Kumar 2014). Therefore, more users
downloading premium apps will be satisfied with their purchase, leading
to less variance.}

In the graph below, we see a significant difference in the variance of
ratings. Premium apps tend to have a higher variance, with more
outliers.

In the graph below, we once again see that level 5 has a high variance
compared to the others. But when adjusted with the Bayesian average, the
variance is one of the lowest. This difference may arise because users
who are satisfied with the app tend to rate the app highly, while users
who are not satisfied, tend to rate it much lower, having paid for the
app.

\subsubsection{Relationships between App
Versions}\label{relationships-between-app-versions}

Level 2 has two version of the same app. A sample and a premium version.
According to hypothesis 3, the rating of a paid version positively
correlates with the rating of the sample version.

\emph{H3: For apps that utilize a sample and a premium version of the
same app (level 2), the rating of the paid-for version is positively
associated with the rating of the free version of the same app. This was
true for the study on the most popular apps in the Google Play Store by
Liu, Au, and Choi (2012), so it is expected a similar pattern should
arise for this dataset.}

In the table below the correlation between ratings of sample and premium
apps is considered moderate positive with 0.35. The correlation between
the Bayesian average rating of sample and premium apps is considered
moderate to strong positive correlation with 0.5.

\subsection{Ethical Consideration}\label{ethical-consideration}

\section{\texorpdfstring{Discussion }{Discussion }}\label{discussion}

\subsection{Reflection on the
Findings}\label{reflection-on-the-findings}

Downloads do not necessarily indicate revenue for freemium models
(Djaruma et al. 2023). The time the user spends on an app and the
purchases made within this app (Ross 2018) are better measures of the
revenue for freemium applications.

\subsection{Practical Implications for
Businesses}\label{practical-implications-for-businesses}

\subsection{Future Research
Directions}\label{future-research-directions}

\section{References}\label{references}

\phantomsection\label{refs}
\begin{CSLReferences}{1}{0}
\bibitem[\citeproctext]{ref-aydingokgoz2021}
Aydin Gokgoz, Zeynep, M. Berk Ataman, and Gerrit H. van Bruggen. 2021.
{``There{'}s an App for That! Understanding the Drivers of Mobile
Application Downloads.''} \emph{Journal of Business Research} 123
(February): 423--37.
\url{https://doi.org/10.1016/j.jbusres.2020.10.006}.

\bibitem[\citeproctext]{ref-bamberger2020}
Bamberger, Kenneth A., Serge Egelman, Catherine Han, Amit Elazari Bar
On, and Irwin Reyes. 2020. {``Can You Pay for Privacy? Consumer
Expectations and the Behavior of Free and Paid Apps.''} \emph{Berkeley
Technology Law Journal} 35: 327.
\url{https://heinonline.org/HOL/Page?handle=hein.journals/berktech35&id=339&div=&collection=}.

\bibitem[\citeproctext]{ref-djaruma2023}
Djaruma, Heryan, Zaki Widyadhana Wirawan, Alexander Agung Santoso
Gunawan, and Karen Etania Saputra. 2023. {``2023 IEEE International
Conference on Communication, Networks and Satellite (COMNETSAT).''} In,
577--83. \url{https://doi.org/10.1109/COMNETSAT59769.2023.10420748}.

\bibitem[\citeproctext]{ref-kumar2014}
Kumar, Vineet. 2014. {``Making {``}Freemium{''} Work.''} \emph{Harvard
Business Review}, May.
\url{https://hbr.org/2014/05/making-freemium-work}.

\bibitem[\citeproctext]{ref-liu2012freemium}
Liu, Charles Zhechao, Yoris A. Au, and Hoon Seok Choi. 2012. {``An
Empirical Study of the Freemium Strategy for Mobile Apps: Evidence from
the Google Play Market.''} In \emph{Proceedings of the 2012
International Conference on Information Systems (ICIS)}, 1--19.
\url{http://aisel.aisnet.org:80/cgi/viewcontent.cgi?article=1050&context=icis2012}.

\bibitem[\citeproctext]{ref-liu2014}
---------. 2014. {``Effects of Freemium Strategy in the Mobile App
Market: An Empirical Study of Google Play.''} \emph{Journal of
Management Information Systems} 31 (3): 326--54.
\url{https://doi.org/10.1080/07421222.2014.995564}.

\bibitem[\citeproctext]{ref-mileros2024}
Mileros, Martin D., and Robert Forchheimer. 2024. {``Free for You and
Me? Exploring the Value Users Gain from Their Seemingly Free Apps.''}
\emph{Digital Policy, Regulation and Governance} ahead-of-print
(ahead-of-print). \url{https://doi.org/10.1108/DPRG-01-2024-0009}.

\bibitem[\citeproctext]{ref-global2019}
{``Mobile App Revenue Worldwide by Segment (2019-2027).''} 2023.
Statista.
\url{https://www.statista.com/forecasts/1262892/mobile-app-revenue-worldwide-by-segment}.

\bibitem[\citeproctext]{ref-nieborg2016}
Nieborg, David B. 2016. {``Free-to-Play Games and App Advertising: The
Rise of the Player Commodity.''} In. Routledge.

\bibitem[\citeproctext]{ref-charted2023}
Richter, Felix. 2023. {``Charted: There Are More Mobile Phones Than
People in the World.''} World Economic Forum.
\url{https://www.weforum.org/stories/2023/04/charted-there-are-more-phones-than-people-in-the-world/}.

\bibitem[\citeproctext]{ref-roma2016}
Roma, Paolo, and Daniele Ragaglia. 2016. {``Revenue Models, in-App
Purchase, and the App Performance: Evidence from Apple{'}s App Store and
Google Play.''} \emph{Electronic Commerce Research and Applications} 17
(May): 173--90. \url{https://doi.org/10.1016/j.elerap.2016.04.007}.

\bibitem[\citeproctext]{ref-ross2018}
Ross, Nicholas. 2018. {``Customer Retention in Freemium Applications.''}
\emph{Journal of Marketing Analytics} 6 (4): 127--37.
\url{https://doi.org/10.1057/s41270-018-0042-x}.

\bibitem[\citeproctext]{ref-salehudin2021}
Salehudin, Imam, and Frank Alpert. 2021. {``No Such Thing As A Free App:
A Taxonomy of Freemium Business Models and User Archetypes in the Mobile
Games Market,''} December.
\url{https://papers.ssrn.com/abstract=4001100}.

\end{CSLReferences}




\end{document}
